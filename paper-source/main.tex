% mnras_template.tex 
%
% LaTeX template for creating an MNRAS paper
%
% v3.0 released 14 May 2015
% (version numbers match those of mnras.cls)
%
% Copyright (C) Royal Astronomical Society 2015
% Authors:
% Keith T. Smith (Royal Astronomical Society)

% Change log
%
% v3.0 May 2015
%    Renamed to match the new package name
%    Version number matches mnras.cls
%    A few minor tweaks to wording
% v1.0 September 2013
%    Beta testing only - never publicly released
%    First version: a simple (ish) template for creating an MNRAS paper

%%%%%%%%%%%%%%%%%%%%%%%%%%%%%%%%%%%%%%%%%%%%%%%%%%
% Basic setup. Most papers should leave these options alone.
\documentclass[fleqn,usenatbib]{mnras}

% MNRAS is set in Times font. If you don't have this installed (most LaTeX
% installations will be fine) or prefer the old Computer Modern fonts, comment
% out the following line
\usepackage{newtxtext,newtxmath}
% Depending on your LaTeX fonts installation, you might get better results with one of these:
%\usepackage{mathptmx}
%\usepackage{txfonts}

% Use vector fonts, so it zooms properly in on-screen viewing software
% Don't change these lines unless you know what you are doing
\usepackage[T1]{fontenc}
\usepackage{ae,aecompl}


%%%%% AUTHORS - PLACE YOUR OWN PACKAGES HERE %%%%%

% Only include extra packages if you really need them. Common packages are:
\usepackage{graphicx}	% Including figure files
\usepackage{amsmath}	% Advanced maths commands
\usepackage{amssymb}	% Extra maths symbols
\usepackage{hyperref}
\usepackage{enumerate}


%%%%%%%%%%%%%%%%%%%%%%%%%%%%%%%%%%%%%%%%%%%%%%%%%%

%%%%% AUTHORS - PLACE YOUR OWN COMMANDS HERE %%%%%

% Please keep new commands to a minimum, and use \newcommand not \def to avoid
% overwriting existing commands. Example:
%\newcommand{\pcm}{\,cm$^{-2}$}	% per cm-squared

%%%%%%%%%%%%%%%%%%%%%%%%%%%%%%%%%%%%%%%%%%%%%%%%%%

%%%%%%%%%%%%%%%%%%% TITLE PAGE %%%%%%%%%%%%%%%%%%%

% Title of the paper, and the short title which is used in the headers.
% Keep the title short and informative.
\title[Mergers in COSMOS]{Mergers in the COSMOS Field}

% The list of authors, and the short list which is used in the headers.
% If you need two or more lines of authors, add an extra line using \newauthor
\author[D. O'Ryan et al.]{
David O'Ryan$^{1}$\thanks{E-mail: d.oryan@lancaster.ac.uk}
\\
% List of institutions
$^{1}$Lancaster University \\
}

% These dates will be filled out by the publisher
\date{Accepted XXX. Received YYY; in original form ZZZ}

% Enter the current year, for the copyright statements etc.
\pubyear{2023}

% Don't change these lines
\begin{document}
\label{firstpage}
\pagerange{\pageref{firstpage}--\pageref{lastpage}}
\maketitle

% Abstract of the paper
\begin{abstract}
   Abstract will go here.
\end{abstract}

% Select between one and six entries from the list of approved keywords.
% Don't make up new ones.
\begin{keywords}
keyword1 -- keyword2 -- keyword3
\end{keywords}
%%%%%%%%%%%%%%%%%%%%%%%%%%%%%%%%%%%%%%%%%%%%%%%%%%

%%%%%%%%%%%%%%%%% BODY OF PAPER %%%%%%%%%%%%%%%%%%

\section{Introduction}\label{introduction}
\noindent Galaxy interaction and merging is of fundamental importance to our current theories of galaxy formation and evolution. Our current cosmological theory, $\Lambda$CDM dictates that structure is expected to form and assemble hierarchically \citep{1978MNRAS.183..341W, 2013MNRAS.436.1765M}. Thus, understanding the underlying processes that occur in galaxy interaction and merging are paramount to understanding cosmological theories. It is well known that many underlying processes are triggered or enhanced due to galaxy interaction. The clearest result of interaction is the disruption of galactic morphology, the acceleration of disks to spheroidal systems and the formation of tidal features \citep{1972ApJ...178..623T, 1977ApJ...212..616T, 2005MNRAS.357..753G, 2009MNRAS.397..802H}, increases in star formation rates when compared to the general galaxy population \citep{1991ApJ...370L..65B, 2006ApJ...652...56B, 2014MNRAS.437.2137S, 2015ApJ...807L..16K} and, potentially, their rapid quenching \citep{2013MNRAS.430.1901H, 2023RAA....23i5026D}. There is also evidence to suggest that interaction plays a role in the activation of active galactic nuclei (AGN) \citep{2011MNRAS.418.2043E, 2015ApJ...806..219C, 2023MNRAS.523.4164H}. Therefore, with interaction having such an affect on the galaxies we see - over a long time period - understanding galaxy interaction is fundamental to understanding galaxy evolution

To better our understanding of galaxy interaction, it is often prudent to break them down into different stages of interaction. This relates to the dynamical time of the interaction, and the snapshot in the interaction we are observing. Previous works focused on simulations find that a sudden increase in star formation occurs early in the dynamical timescale of the interaction \citep{2008MNRAS.384..386C, 2019MNRAS.490.2139R}, however, confirming this definitive link observationally remains elusive \citep{2023arXiv230916531R}. The same studies could also be done for nuclear activation as well as quenching of interacting systems \citep{Papers on nuclear activation}. While we cannot yet directly measure when in the dynamical timescale star formation is enhanced from observations, we can approximate it by splitting different interactions into different stages. These stages relate directly to the dynamical timescale of the interaction. Often, they are split into four: gravitationally bounded pairs, approaching pericentre of the interaction, approaching apocentre of the interaction, and approaching coalescence.
 
To fully study the relationship between interaction stage and the underlying physical processes occurring we require large, statistically robust interacting galaxy samples. However, they are surprisingly difficult to create; with the first interacting galaxy catalogue (the \citet{1966ApJS...14....1A} catalogue) containing only 318 systems. More recent samples have been created either by machine learning classification \citep{2019A&A...626A..49P, 2023A&A...669A.141S}, visual classification by citizen scientists \citep{2010MNRAS.401.1043D} or by photometric parameterisation \citep{2004AJ....128..163L, 2023MNRAS.522....1N}. However, each sample has always been plagued by contamination and a loss of statistical significance when broken down into stages. It is notoriously difficult, from morphology alone, to select interacting or merging galaxies. Without redshift information it is difficult to discern if they are close together in 3D space or only by projection effects.
 
We elect to use the newly created interacting galaxy catalogue created by \citet{2023ApJ...948...40O} (hereafter, OR23) which found a sample of 21,926 interacting galaxy systems. This is the largest visually classified interacting galaxy catalogue to date. This sample was created using the new ESA Datalabs\footnote{ESA Datalabs: link} platform and the convolutional neural network machine 
\texttt{Zoobot} \citep{2022MNRAS.509.3966W, 2023JOSS....8.5312W}. This classified all sources in the Hubble Source Catalogue \citep{2016AJ....151..134W} from the Hubble Space Telescope (HST) archives into interacting or non-interacting. Once this was done, a rigorous process of de-duplication, contamination removal and visual inspection was conducted.

In this work, we cross match this catalogue with the Cosmic Evolutionary Survey (COSMOS) survey\footnote{DOI: \href{https://www.ipac.caltech.edu/doi/irsa/10.26131/IRSA178}{10.26131/IRSA178}}. This provides us with ancillary data for a subset of the OR23 catalogue. This ancillary data contains many galactic parameters of interest and we explore how they evolve with different stages of interaction. We define four distinct stages that occur over the course of the full dynamical timescale of interaction, from the two galaxies approaching each other to full coalescence. We primarily focus on the evolution of the stellar masses, star formation rates, surrounding environment and AGN classification from various COSMOS catalogues across the full dynamical timescale of interaction. With the photometric redshifts available from COSMOS, we also confirm a set of interacting galaxies and close pairs to investage further parameters of interest. These include the trend of projected separation of the galactic pairs on star formation rate as well as the change with mass ratio.

%, comparing to the work of \citet{2022ApJ...940....4S} a different study using COSMOS to investigate these latter relations.

This paper is laid out as follows in Section \ref{data} we explain the OR23 and COSMOS catalogues, and describe the process of catalogue matching especially focused on accurate de-duplication and cross-matching selection from the different catalogues.  Section \ref{staging} describes the methods by which we split the interacting systems into their different stages, and how we identify the secondary galaxies to each identified interacting galaxy. We also discuss our processes of calculating projected separations. In Section \ref{results} we describe and discuss our results, and their implications, and compare to previous works focused on investigating different stages of interaction. Finally, Section \ref{conclusion} we make concluding remarks and discuss future work to better our constraints. 

Where necessary, we use a Flat $\Lambda$CDM cosmology with $H_0$ = $70$\,km/s/Mpc and $\Omega_M = 0.3$. Hereafter in this paper, when referring to an interacting galaxy we are referring to a galaxy which has undergone one or multiple flybys by a secondary galaxy and caused tidal disturbance. A merging galaxy is the final state of these flybys, where two or more systems have or are about to coalesce to form a highly morphologically irregular system.

\section{DATA: Catalogue Matching \& Secondary Identification} \label{data}
\subsection{The O'Ryan+23 Catalogue}
\noindent We utilise the interacting galaxy catalogue created in OR23. While the full process of its creation is detailed there, we briefly describe it here. This sample of interacting galaxies was created using a convolutional neural network - \texttt{Zoobot} - to search all final product data in the \textit{Hubble} Legacy Archive. A regular convolutional neural network classifies images by extracting features from them, and using this to output a solution. \texttt{Zoobot} differs by extracting features from an image, and then reduces them to a generalised representation (essentially, a small feature vector). The power of this is that it makes \texttt{Zoobot} incredibly generalisable. This is further reinforced by training it on citizen scientist responses to the Galaxy Zoo: DECaLS \citep{2022MNRAS.509.3966W} project to every task in that project. Thus, it can answer a host of questions as well as be very easily finetuned to new datasets and classification tasks. In OR23, it was easily finetuned on \textit{Hubble} data by using classifications from all previous Galaxy Zoo projects \citep{2008MNRAS.389.1179L}.

Cutouts of each source were created as single-band \textit{F814W} images at 150 $\times$ 150 pixels across. This choice was motivated by it having the highest number of final product FITS files associated with it, while the image size was kept small due to the high number of cutouts required to be stored. Source coordinates were found in the \textit{Hubble} Source Catalogue \citep{2016AJ....151..134W}. This corresponded to just over 92 million cutouts to create and sources to classify. Due to issue of classifying interacting galaxies by morphology alone an extremely conservative prediction score from \textit{Zoobot} was required. This resulted in a very pure, but highly incomplete catalogue of 21,926 interacting galaxy systems. This is the largest catalogue of interacting galaxies directly found to date.

However, in the OR23 catalogue, only the coordinates and the \textit{Hubble} Source catalogue are provided. For ancillary data, this catalogue must be matched to different data repositories. This is not a trivial task, as a large proportion of the OR23 catalogue is known to not exist in any existing astrophysical data repository. In this work, we opt to match the OR23 catalogue to the COSMOS catalogues.

\subsection{The COSMOS2020 Catalogue}
We opt to match the OR23 catalogue with the catalogues of the COSMOS survey. Specifically, we elect to use the COSMOS2020 catalogue \citep{2022ApJS..258...11W}. This catalogue has a wealth of ancillary information for just over 1.7 million sources in a 2 square degree area of the sky. Each source found in this area has been analysed with well used astronomical software, for our purposes primarily LePhare \citep{1999MNRAS.310..540A, 2006A&A...457..841I} and EAzY \citep{2008ApJ...686.1503B}. These provide two different measurements of the parameters we wish to investigate and match with stage. The first is the stellar masses of each our cross matched galaxies and the estimated star formation rates. We opt to simply use one of these two measurements in our analysis. For the stellar mass we use the LePhare measurement and for the star formation rates we use the EAzY measurements. Using either the vice versa of this, both EAzY or both LePhare measurements led to the degeneracy's within the underlying model Spectral Energy Distribution (SED) templates being revealed. Therefore, using these two opposing parameters gave realistic approximations of these photometrically approximated values. 

As stated previously, the OR23 catalogue only contained the source ID, right ascension and declination of each interacting galaxy. Therefore, we had to cross match these with the COSMOS2020 catalogue right ascension and declination measurements. To do this, we searched the COSMOS2020 catalogue for any sources within 10" of the OR23 sources. If more than one COSMOS2020 source was within this radius, we took the closest source to the OR23 coordinates as the correct source. Upon matching based on coordinates, we conducted a de-duplication of the resultant catalogue based on COSMOS2020 ID, and reassigned COSMOS2020 sources to each duplication. If no further COSMOS2020 sources were within 10" of the OR23 source, then this was considered not to be cross matched. We find that, after this cross matching and de-duplication, 3,786 of the OR23 sources exist in the COSMOS2020 catalogue.

We make two further cuts, based on the ancillary data available in the COSMOS2020 catalogue. First, there are many unphysical photometric measurements for the stellar mass or star formation rates. Therefore, we cut out these results as spurious. We only keep systems within a mass range of $6.5 \leq \log_{10} \text{M}_{*}(\text{M}_{\odot}) \leq 12.5$ and a star formation rate range of $-5 < \log_{10} \text{SFR} (M_{\odot}yr^{-1}) \leq 3.5$. Secondly, we institute a redshift cut of $z \leq 1.2$. This is to match the catalogue that will be later used to investigate the incidence of our sample of interacting galaxies with environment (detailed in Section \ref{data:environ}). Applying these cuts reduces our sample size to 3,689 interacting galaxies.

\subsection{Secondary Identification}
\noindent As stated previously, the OR23 catalogue only identified interacting galaxy systems, not individual interacting galaxies. Therefore, we must find the secondary in each interaction using the COSMOS2020 catalogue. The size of the cutouts in this catalogue were also small, and therefore, the secondary did not often appear in the cutout in that project. To find the secondaries, we apply three steps. First, a cutout surrounding the primary of each system was created. These cutouts were from the COSMOS cutout service, selecting HST-ACS tiles as the source of the cutouts in the $F814W$ filter. Each cutout had a 30" radius (corresponding to 1001 $\times$ 1001 pixels) centred and was centred on the primary galaxy. The original OR23 cutout was also displayed next to the enlarged cutout. We then searched the COSMOS2020 catalogue for all sources within the cutout. Their positions and redshifts were recorded and used to annotate the cutout.

We then visually assessed each annotated cutout, giving one of the four following classifications to each: system disturbed but secondary could not be identified; secondary could be identified; cannot confirm galaxy is interacting; null redshift; incorrect primary assigned. Our definition of a secondary galaxy was such that the secondary had to be within the cutout we were visually assessing and within $z=\pm0.1$ of the primary redshift. We use this cutoff as this is often used when calculating environment parameters \citep[e.g][]{Baldry 2006} or defining interacting galaxies by close pairs \citep[e.g][]{Shah et al. paper from 2022 / 2023} based on the uncertainties present in using photometric redshift estimates from SED fitting. We define a null redshift as one outwith our redshift limits or NaN. A minority of the cutouts we visually assessed were found to have the incorrect primary at the centre. This was a result of our cross matching based on radius with the COSMOS catalogue. 

Using these definitions, we find that of the 3,689 original systems cross-matched with COSMOS2020 2,209 could not have their secondary identified, 834 had a clear secondary, 446 could not be reliably classified as an interacting galaxy, 248 had a null redshift and 149 were the incorrect primary. Figure \ref{fig:secondary_selection} shows an example of each of our classifications. Each secondary we identify was added to our sample, increasing our sample size to 3,829.

\begin{figure*}
    \centering
    \includegraphics[width=\textwidth]{figures/cutouts_ex.pdf}
    \caption{An example of each visual classification made on the cross matched sample. From left to right these are the secondary could not be identified, the primary had a clear secondary, the primary could not be reliably classified as an interacting galaxy, the redshift was null and the incorrect primary was identified in OR+23. Based on these classifications, we either add the secondary galaxies to the sample or we remove the contamination from it. These images are 30" across using the COSMOS cutout service, selecting HST/ACS tiles as the basis for the observations in the $F814W$ filter.}
    \label{fig:secondary_selection}
\end{figure*}

% Define our classifications above.
While initially surprising that the majority of our systems could not have a secondary identified, we found that it was mostly due to limitations in the COSMOS catalogue or the way in which we had conducted our secondary identification. To define a secondary galaxy, the object had to have a ID associated with it in the COSMOS2020 catalogue. However, when two systems were very close together and small enough, often they would be identified under a single COSMOS ID despite being two separate systems. The same also occurred when two systems were merging or interacting. The tidal features connecting the systems would lead to only one of them being identified in the catalogue, or when two cores appeared in one merging system. Figure \ref{fig:secondary_selection} specifically an example of two systems being close enough together that they have been identified under a single COSMOS2020 entry. Figure \ref{fig:secondaries_found} shows this disparity with the different types of interaction we observe.

\begin{figure}
    \centering
    \includegraphics[width=0.45\textwidth]{figures/visualisation_classification.pdf}
    \caption{Where a secondary could be identified at different stages in the interaction. The reasoning for such disparity in secondaries identified is due to the relative distance each the secondary would be from the primary at each stage. For a close pair, we often found the secondary galaxy, but a minority of these were so close together that the entire system was given a single COSMOS ID. The same was true for those interacting systems approaching pericentre or merging. When the secondary was near apocentre, often it would be outwith the cutout we were using for visual classification.}
    \label{fig:secondaries_found}
\end{figure}

Those galaxies which were found to be contamination (i.e., could not be reliably classified as interacting or having a null redshift) were removed from our sample. Galaxies that could not be reliably classified as interacting were those which showed little tidal distortion and had no systems nearby with a redshift approximately that of the primary. Often, these were systems that were overlapping but at different redshifts, systems with irregular morphologies of spiral arms or clumpy galaxies. There was also many systems that were at high redshift ($z \geq 1$) where the resolution of the cutouts meant that features could not be discerned visually.

Finally, we found that there were 149 systems which had been assigned to an incorrect primary. These were systems where the interacting galaxies were clearly in the cutout but some tidal debris or some nearby system had been assigned as the primary instead. This was a result of our catalogue matching based solely on separation between the OR23 coordinates and the COSMOS2020 source coordinates. We reassign these systems to the correct COSMOS IDs and then take them through the secondary identification process.

By the end of this pipeline, we find a sample size of 3,829 interacting galaxies. We conduct a de-duplication based on the COSMOS2020 ID, which reduces the sample back down to 3,547 interacting galaxies. This shows that many of the secondaries we had identified for the different systems were already found in the OR23 catalogue. The remaining systems are each visually confirmed interacting and disturbed galaxies based on their morphology and photometric redshift as measured in the COSMOS2020 catalogue.

\subsection{Finding Additional Systems}
\noindent As a result of using visual classification to find the secondary galaxy in each interaction, we were able to also confirm other interacting systems which had not been found in the OR23 catalogue. Primarily, these extra interacting galaxies were more systems that were involved in the interactions already recorded in OR23. The pipeline described previously was only used to find a secondary for each interacting galaxy, however in many cases more galaxies were involved. Other interacting galaxies that were added here were low redshift systems which would have appeared very large in the OR23 cutout. Details and tidal features were then lost in the original classification process. By looking at the larger COSMOS2020 cutouts used here, we were able to confirm these were interacting galaxies and add them to the catalogue. The final set of extra systems we added were often those that were close pairs, but physically in 3D space. While they had little tidal disruption - what the interacting galaxies were classified on in OR23 - with the photometric redshifts to hand, we could correctly identify them here.

In total, we found an extra 841 interacting systems that we could add to our sample. Upon conducting a de-duplication of these with the sample already found, this was reduced to 634 interacting systems. This gives us a total sample size of 4,181 which we use through the rest of this work.

\section{METHOD: Environment, AGN and Interaction Stage}
In this section, we describe how we separate our sample of 4,181 interacting galaxies into separate stages. Each stage is defined to capture a different part of the interaction, and a different part of the dynamical history of the interaction. We want to compare different stages in the interaction to different physical underlying processes, such as star formation rate. Such parameters can be found in the COSMOS2020 catalogue. However, two fundamental properties we also want to compare with is that of AGN fraction evolution with time and that of environment, and the impacts this might have.

The parameters required to calculate the AGN fraction are not found in the COSMOS2020, and we therefore must cross match with other catalogues to find the required parameters. We also describe the catalogue we use here to define the density parameter around each of our sources, and define how the environment is calculated there.

\subsection{Classifying Stage of Interaction}\label{sec:staging}
The primary goal of this work is to find if a relation exists between a host of galactic parameters, underlying physical processes and the stage of the interaction. We, therefore, need to define different stages for them. We have hinted at this in the previous section with Figure \ref{fig:secondaries_found}. We break our sample down into four stages:

\begin{itemize}
    \item Stage 1: Systems which are well-separated with little to no morphological disturbance.
    \item Stage 2: Close pairs showing either morphological distortion or physical connection by tidal features.
    \item Stage 3: Well separated pairs with morphological disturbance or isolated galaxies with clear tidal features.
    \item Stage 4: Highly disturbed systems with two or more cores within.
\end{itemize}

\noindent Examples of each stage can be found in Figure \ref{fig:stages}. Each stage is designed to capture a different part of the interaction history of a system in the total dynamical time.

\begin{figure}
    \centering
    \includegraphics[width=0.45\textwidth]{figures/examples-stages.pdf}
    \caption{Examples of the four stages we split our interacting galaxy sample into. Stage 1: A close pair with confirmed redshift matching. Stage 2: Two distinct systems interacting with tidal features forming. Stage 3: A tidally disturbed system with no secondary present, likely at apocentre. Stage 4: A galaxy with multiple cores while highly disturbed. At the final stage before coalescence.}
    \label{fig:stages}
\end{figure}

Stage 1 of the interaction captures the first approach of the two systems. This is the time when the two galaxies potentials are beginning to have an effect on one another. They have no morphological disturbance yet and, in our sample, are most often those galaxies with distinct disks. Hence, this stage has the highest number of if identified secondaries with every system having a secondary that can be visually classified. In this sample, we would expect no effects from the interaction to have began yet. 

Stage 2 of the interaction is defined as the point where the two galaxies in the interaction are approaching or just passing the pericentre of the tidal encounter. For those galaxies before pericentre, we would expect the beginning of morphological disturbance or some irregularity in the two systems. Once the galaxies have passed pericentre, we would define it by much higher morphological disturbance, distortion and tidal debris. This requirement for morphological disturbance is how we differentiate stage 2 from stage 1 interacting galaxies. Throughout this stage of the interaction, in our sample, we see significant overlap or connection of the two systems in the form of tidal features and morphological distortion.

Stage 3 interacting galaxies are described as those interacting systems where the systems have approached apocentre from each other. In many of cases of this stage, if the galaxies have sufficient velocity, the galaxies will escape each other and not coalesce. Hence, galaxies at this stage often could not have a secondary galaxy identified in our visual assessment. We purely describe a galaxy as stage 3 based on significant tidal and morphological disturbance. This stage has the highest potential for contamination by irregular galaxies or galaxies whose morphology has been disturbed by processes other than tidal interaction. We discuss the potential impacts of this contamination, and the limitations of using visual morphology as a proxy for stage in Section \ref{sec:limitations}. 

Finally, stage 4 represents the final step of a galaxy interaction. If the two galaxies do not have sufficient velocity to escape one another, then they go on to coalesce and ultimately merge. We define this stage such that the sample is of galaxies just before this coalescence and the two cores remain distinct. However, it is important to note that this stage may be contaminated by some post-merger remnants. In this context, we define a stage 4 galaxy as one with significant morphological disturbance but with multiple cores present in the disks. At this stage, the interaction is at it's most violent with complete disruption of both galactic disks. We also expect star formation rate to increase significantly in this stage. This is important to note, as high star forming at specific redshifts in the $F814W$ filter may appear like a second core within a galactic disk. At this point, we would be observing a post-merger remnant rather than the point just before coalescence. Therefore, to mitigate potential contamination in from the post-merger stage we only classify stage 4 galaxies as those with highly disrupted disks and second cores close to the primary core (and not far out in the disk). 

Our four stage approach is not a new one, and many other works have utilised it to differentiate different parts of the dynamical history of a galaxy interaction \citep{}. It is, however, important to note its limitations. First, this is a simplified view of the dynamical timescale of an interaction. We have assumed that an interaction begins, the pair have enough velocity to escape or they coalesce and that is representative of the full dynamical timescale. We have not considered systems where there are multiple flybys before coalescence or systems where there are multiple members at different stages of the interaction. This is a particular concern when looking at stages 2 and 3. For instance, the effects could be enhanced or diminished over multiple flybys, and we have no way to capture this.

However, with these simplifications in our assumptions, we will investigate the parameters which are proxies for the underlying physical processes and interpret them. We will investigate the whether we see have a bias in our systems based on environment, whether we find increased AGN activity and if we observe enhancement in star formation with stage. The star formation rate is already present in the COSMOS2020 catalogue from both EAzY and LePhare. For the environment parameter and whether AGN are detected, we cross match to two further catalogues. We describe these catalogues here, our process of cross matching and the new sample sizes we have in each subsample.

\subsection{Matching to Environment Catalogue}\label{data:environ}
\noindent There is no measure of the environmental density in the COSMOS2020 catalogue. Such a measure is often calculated in numerous ways, such as the N-nearest neighbour \citep{2006MNRAS.373..469B}, different Bayesian metrics \citep{2008ApJ...674L..13C} or estimating it from Voronoi Tesselation \citep{2021inas.book...57V}. However, in this work, we utilise the existing environmental density catalogue produced by \citet{2017ApJ...837...16D}. This catalogue was created specifically for the COSMOS survey, and has a measured density for all sources with mass $\log(\frac{M}{M_\odot}) \geq 9.6$ and out to a redshift of 1.2. In this catalogue, they calculate not only the density, but also the density parameter $\delta$ and assign each source to a field, filament or cluster classification. For a full description of how they calculate the environment and density field see \citet{2015ApJ...805..121D} and \citet{2017ApJ...837...16D}, but we will briefly describe it here.

To build the density field throughout the COSMOS field, they first construct a set of overlapping redshift slices. Within each slice, a subset of the galaxies are selected such that the median of the probability distribution function (PDF) of their photometric redshift is within it. Then, from this subset, they calculate the weighted surface density within the redshift slice. The weighting is based upon the PDF of the photmetric redshift present within the redshift slice. These weights significantly reduce the effects of projection effects. They then apply a weighted adaptive kernel smoothing using a 2D Gaussian kernel whose width changes based on the found local density of galaxies. Once this density field is created, the density around the sample galaxies can simply be interpolated across the density field based on the angular position and the redshift slice the sample galaxy is in (based on its photometric redshift PDF).

The result of this process, and the cuts defined previously, is a catalogue of $\approx$45,000 galaxies with their densities accurately measured. As advised with the catalogue, we remove any sources which are flagged as uncertain. This is recorded in the catalogue. This leaves us with $\approx$39,000 sources with which to cross match the OR23 sample. We apply the same cuts to our sample as applied in \citet{2017ApJ...837...16D}, and only consider those systems with a mass $\log(\frac{M}{M_\odot}) \geq 9.6$. To cross match with our sample, we simply utilise the COSMOS2015\_ID which is recorded in the COSMOS2020 catalogue which corresponds to the ID in the \citet{2017ApJ...837...16D} catalogue. We find that every source in our COSMOS2020 sample has a COSMOS2015 ID associated with it, and therefore we fully match the reduced catalogue.

Upon applying the mass cut to our sample, we find that we are matching 2,800 sources to the \citet{2017ApJ...837...16D} catalogue. Upon matching based upon the COSMOS2015 ID, we find that 628 sources in our sample do not exist in the environment catalogue. This reduces our sample to 2,172 galaxies with confirmed and reliable environment density measurements.

\subsection{Classifying AGN}
\noindent We will also investigate the affect of interaction stage upon the AGN activity within the galaxies involved. Unfortunately, the outputs of EAzY and LePhare do not provide us with photometric estimates of the line fluxes needed to define an AGN on a Baldwin, Philips and Terlevich (BPT) diagram \citep{}. Therefore, we cannot directly identify if a galaxy is only star forming, has nuclear activity or is a composite. So, we turn to cross-matching our sample to three different catalogues which contain the required information. We use the Chandra COSMOS Legacy Survey Multiwavelength Catalogue \citep{2016ApJ...817...34M} and the COSMOS VLA 3GHz survey \citep{2017A&A...602A...6S, 2017A&A...602A...3D}. Each source has an AGN, star forming or composite classification is within the catalogue.

% Need a small paragraph here about these two catalogues and their differences.

Again, our cross match process is very much the same as previously described. We use the previously identified COSMOS2020 source coordinates and conduct a radial search of 10". If more than one source is within this radius, we match on the closest source. The two catalogues derived from the COSMOS survey do not contain IDs which overlap with the COSMOS2020 catalogue, therefore we must do it based on coordinates. We conduct our matching in serial, starting from the \citet{2017A&A...602A...6S} catalogue and then to the Chandra survey. At every step, if we find a match in the relevant catalogue, we remove it from subsequent searches in other catalogues and take the first classification as the correct one.

Applying our matching criteria, we find 1,059 classifications in the VLA 3GHz survey and 155 in the Chandra survey. These were split into 812 star forming galaxies and 402 AGN. We also investigate cross matching with the MPA-JHU catalogue \citep{2003MNRAS.341...33K, 2004MNRAS.351.1151B, 2007ApJS..173..267S}, however, found that all matches were already represented by the VLA and Chandra surveys. Therefore, we only utilise these two catalogues in our sample.

While the ratio of AGN to star forming galaxies in our sample seems large compared to other works, it is important to note that this is a result of limited matching between the catalogues. Of the 4,182 galaxies in our sample, only 1,214 appeared at all in either the VLA or Chandra catalogues. Any galaxy which did not appear in these catalogues, we discard as unclassified. Therefore, we require these catalogues to specifically show us if a galaxy is star forming or not before we classify it as such. This is why our number of star forming galaxies is lower than expected. We discuss this limitation in section \ref{sec:limitations} and the effect it may have on our results of AGN fraction evolution with stage.

\section{STAR FORMATION EVOLUTION WITH INTERACTION STAGE}\label{results}
\subsection{Controlling for Interaction Stage}
\noindent With our samples selected we investigate the change in multiple parameters with stage of the interaction. As stated previously, the four stages of interaction are designed to capture the main different parts of galaxy interaction. First, we show the results of breaking down our sample into stages with relation to the star formation and mass of our sample. We utilise the estimates of these parameters that exist in the COSMOS2020 catalogue itself. We have the option to use either the measured estimates from LePhare or EAzY. We opt to use the stellar mass measured from LePhare and the star formation rate measured with EAzY for our final results. We choose this as using any other combination reveals the degeneracys of the SED template fitting conducted in either of the underlying softwares. Figure \ref{fig:difference-measures} shows how the two measures differ from each other. As shown, EAzY measures larger stellar masses compared to LePhare while the scatter in SFR is very large.

\begin{figure}
    \centering
    \includegraphics[width=0.45\textwidth]{figures/mass-sfr-scatter.pdf}
    \caption{Comparison of the measures of stellar mass and SFR using either LePhare or EAzY photometric codes to calculate them. \textit{Left}: The scatter in the stellar masses between softwares. As shown, EAzY often seems to find larger stellar masses when compared to LePhare. \textit{Right}: Scatter in SFRs when measured with LePhare or EAzY. There is a large difference between the measures of LePhare and EAzY here, and therefore we opt to only use EAzY measures of the SFR through this work.}
    \label{fig:difference-measures}
\end{figure}

As the measure of the masses between the softwares is similar, we can interchange which mass we use in these measurements. However, with the very large scatter of the SFRs between the softwares, we opt to only use the EAzY measurement. % Where do these differences come from? What potential impact could this have on our results?

With the measures of stellar mass and SFR selected, we break down our sample into their different stages. Figure \ref{fig:sfr-mass} shows the breakdown of these parameters with stage. On a population scale, there is clear evolution in the star formation rate from stage 1 through to stage 4. In stage 1, where the galaxies are distinct from one another with no clear morphological disturbance, we clearly see two populations of galaxies. These are the blue, star-forming and red sequences. The contours of number density in Figure \ref{fig:sfr-mass} bring out this population. In stage 2, when the galaxies are at pericentre, this red sequence remains but is highly diminished while the star forming sequence remains constant. Stage 3 shows a similar effect, where the red sequence reduces again before finally in stage 4 the red sequence completely disappears.

\begin{figure}
    \centering
    \includegraphics[width = 0.45\textwidth]{figures/sfr-mass-stages.pdf}
    \caption{The LePhare stellar mass against the EAzY SFR across the different stages of the interaction. The blue contours are 8 levels of density of the underlying populations in each frame. \textit{Top left}: The stellar mass and star formation rate of Stage 1 of our sample. Here, the interacting galxies are simply close pairs with little to no morphological disturbance. There are clearly two populations here: a main, star forming sequence forming the main population and a smaller red sequence. \textit{Top right}: Stage 2 of the interaction, where the two interactors are close to pericentre. The star forming sequence remains, but the red sequence is reduced significantly. \textit{Bottom left}: Stage 3 of the interaction, where the interactors are close to apocentre or escaped. Here, we see the almost complete disappearance of the red sequence. \textit{Bottom Right}: Stage 4 of the interaction, where the two systems are close to coalescence or have merged. The red sequence of galaxies has completely disappeared.}
    \label{fig:sfr-mass}
\end{figure}

However, this result could also be due to many other factors rather due to interaction stage. For instance, if the mass distribution of our sources evolves as well, it could just be that we are selecting higher mass systems in stage 4 when compared to stage 1. This would have the result of systems in stage 4 having, on average, higher SFRs than those in stage 1 and appearing like we had evolution in the star forming population with stage. Another effect that could cause this relation to appear would be our selection was highly dependent on galactic environment. It is well known that environment and star formation are closely linked, and that classifying interacting galaxies by morphology classifiers can weight up the cluster environment over galaxies in the field. In Section \ref{sec:env-cont}, we control for the environment in our sample and show that this is not an issue. Here, we investigate the question of the evolution of the mass distribution that could give us this result.

We can quantify the similarity of the mass distributions, and then the SFR distributions, using well known statistical tests. We opt to use two different tests: the KS-test and the Shapiro-Wilkes test. The KS-test is excellent at comparing different weighted distributions to each other, while the Shapiro-Wilkes test compares these distributions to other well known functions (in our case, the normal function). Thus, by weighting the mass distributions and using a KS-test, we will be able to quantify how alike the mass distributions are and then compare this to the weighted distribution of the SFRs. The weighting scheme we utilise normalises the distributions such that each bin in the distribution had the same number of sources within it. Therefore, any bins with few than the average number of sources will be weighted up while those bins with more will be weighted down. These weights applied to the mass distribution are then applied to the SFR distribution so the counts in each bin are, essentially, the same.

\begin{figure}
    \centering
    \includegraphics[width = 0.45\textwidth]{figures/stellar-mass-dist.pdf}
    \caption{The stellar mass distribution across the four stages. Each bin is weighted to have equivalent counts within them.}
    \label{fig:weighted-mass}
\end{figure}

\begin{figure}
    \centering
    \includegraphics[width=0.45\textwidth]{figures/sfr-dist.pdf}
    \caption{SFR distribution weighted by mass across each stage.}
    \label{fig:weighted-sfr}
\end{figure}

Figure \ref{fig:weighted-mass} shows the weighted mass distributions through the four different stages. These appear highly similar, and we conduct Kolmogorov-Smirnov \citep[KS-test;][]{an1933sulla} and Anderson-Darling \citep[AD-test;][]{AD_paper} tests to compare each distribution. In each test, we calculate the value of the KS test and a p-value which is the probability that each distribution is drawn from the same parent distribution. For each mass distribution, we find that p-value of the KS test is approximately one. Thus, it is not unreasonable to assume that the galaxies between each of our samples are drawn from the same parent sample and are very similar. Figure \ref{fig:weighted-sfr} shows the SFR distributions while being weighted and controlled for mass. When comparing stages 1 and 2, 1 and 3, 1 and 4, 2 and 4 and 3 and 4, the p-values are $\ll$0.05. Thus, we can reject the null hypothesis for these distributions and assume they are from different parent samples and are not identical. However, for comparing stages 2 and 3, the p-value of the KS test is $\approx$0.90. Thus, while these distributions are likely to be not identical, they are very similar in the parent sample they have been drawn from. \textbf{Thus, for the same mass distribution through each stage of interaction, the star formation distribution radically changes from stage 1 to stage 2 and from stage 3 to stage 4, while remaining reasonably similar from stage 2 to stage 3.}

Putting this result into the context of the dynamical timescale of an interaction, it shows there are three separate stages star formation evolution in these systems. The first is the change from isolated SFRs to interacting SFRs. Here, the SFR between galaxies which are close pairs and those that are at pericentre are radically different. This difference persists through to stage 3 - where the galaxies are at pericentre in the interaction. The SFR distribution remains approximately the same between these two, meaning the forces that drive and affect star formation remain equivalent between these two stages. Finally, the SFR radically changes again when we approach the merging or post-merger stage of the interaction. We can also say that this change is likely an enhancement in the SFR of the galaxies through the interaction due this change being driven by the disappearance of the red sequence through each stage.

We also control for redshift in our examination of this result. While the red sequence is significantly reduced across stage, it is important that we put this into the context of the star forming main sequence of galaxies. We expect that, due to an interaction, the population of starbursting and quiescent galaxies will change with stage. Therefore, we define a star forming main sequence through across redshift. This allows us to classify each source based on its SFR independently of redshift. To define the star forming main sequence (SFMS), we follow the example of \citet{2019MNRAS.484.4360A}. There, they define the star forming main sequence as a function of galactic stellar mass and redshift as

\begin{equation}
    \log \text{SFR}_{\text{MS}}(z)[M_\odot yr^{-1}] = -7.6 + 0.76\log\frac{M_*}{M_\odot} + 2.95\log(1+z).
\end{equation}

\noindent This finds the expected, main sequence, SFR of a galaxy at a given stellar mass, $M_*$, and redshift, z.

The ratio of the measured galaxy SFR and the expected main sequence SFR is then taken. We then use this fraction to classify each galaxy into distinct bins. These are:

\begin{enumerate}[(i)]
    \item Starburst galaxies. Here, the galaxy SFR is highly elevated compared to the SFMS. We follow \citet{2019MNRAS.484.4360A} and define a cutoff of $\log$SFR/SFR$_{\text{MS}} >$ 0.4.
    \item Main sequence galaxy. The galaxy is within 0.4 dex of the SFMS and approximately has the expected SFR. Defined as $-0.4 < \log$SFR/SFR$_{\text{MS}} < 0.4$.
    \item Sub-main sequence galaxy. A galaxy whose SFR is below the majority of the SFMS, but likely not quiescent. Defined as $-1.3 < \log$SFR/SFR $_{\text{MS}} < $ -0.4.
    \item Quiescent (High) galaxy. A galaxy with an SFR in the top $\approx$50\% of the quiscent galaxy population. Defined as $-2.3 < \log$SFR/SFR $_{\text{MS}} <$ -1.3.
    \item Quiescent (Low) galaxy. A galaxy with low SFR and very likely completely quenched. Defined as $\log$SFR/SFR $_{\text{MS}} <$ -2.3.
\end{enumerate}

\noindent We split our sample into its different stages and apply these criteria. Figure \ref{fig:sfr-clsf} shows the ratio between the expected SFMS SFR and the found SFR in COSMOS. This clearly shows a large increase in galaxies classified as starburst from stages 1 to 4 and a large reduction in the number of quenched systems. Figure \ref{fig:sfr-clsf-bar} shows the change in fraction of the different galaxy classifications through stage. These Figures reflect the results found in Figure \ref{fig:sfr-mass}. Initially, in stage 1, we find that the majority of our galaxies lie on the SFMS or just below it. There also exists a small population of galaxies which are classified as a starburst with a population of quiescent galaxies that is roughly double the fraction. As we move through the interaction stage, we see that the quiescent galaxy fraction gradually decreases to the point of almost non-existence in stage 4 galaxies. The complete inverse is true in our starburst fraction. We find that this almost quadruples in the fraction of our sample over the course of the different stages of interaction. The fraction of galaxies on the SFMS remains dominant throughout, however, we do find the fraction of sub-MS galaxies significantly reduced.

\begin{figure}
    \centering
    \includegraphics[width=0.45\textwidth]{figures/sfr-clsf-dist.pdf}
    \caption{Stellar mass against the ratio of measured SFR to the expected SFR if the galaxy was on the SFMS. Black points are the individual sources, while the blue contours are as in Figure \ref{fig:sfr-mass}. The red dotted lines show the cutoffs for different galaxy classifications based on their SFR, with each cut off being defined by the text in blue. We find that through interaction stage, the quiescent galaxy population significantly reduces while the starburst population rapidly increases. As these cutoffs are also dependent on redshift, we find that this evolution in SFR with interaction stage is independent of redshift.}
    \label{fig:sfr-clsf}
\end{figure}

\begin{figure}
    \centering
    \includegraphics[width=0.45\textwidth]{figures/sfr-clsf-bar.pdf}
    \caption{The change in fraction of different galaxy classifications from the fraction of SFR to the expected SFR on the SFMS. While galaxies on the SFMS remain dominant through each sample across interaction stage, there is significant change in the starburst and quiescent populations. The starburst galaxy population moves from being roughly half the size of the quiescent population in stage 1 to completely dominating it in stage 4. This is occurring while the quiescent population is significantly reduced to almost non-existence in stage 4.}
    \label{fig:sfr-clsf-bar}
\end{figure}

% So, what does this all mean?
Thus, we find that throughout the different stages of interaction the star formation within the systems is being driven upwards. This is not an unexpected result. It has long been known that the SFRs of merging and post-merging galaxies is increased with respect to control samples of galaxies within the field \citep{papers saying this}. However, what we have found here is that this process of increasing SFR and galaxies being driven to starburst begins from the pericentre of the first interaction. Simulations have long showed that there is a peak in SFR initially in an interaction, followed by a slow reduction to a final enhanced level in the post-merger stage \citep{papers saying this}. However, due to the initial starburst, it is often shown that the gas within the galaxy is either quickly used up and the galaxy quenches or it is heated to the point that star formation becomes very inefficient. Thus, what we would expect to see here is that stages 2 and 4 would have the highest fractions of galaxies undergoing a starburst and the smallest quiescent fractions.

Our work here is found to be correct by both simulations and previous works on when the starburst occurs in an interacting galaxy. We observe significant change in the starburst and quiescent fractions from stage 1 to 2 and stage 3 to 4. Thus, the initial part of the interaction appears to have a significant impact on the SFR (and, therefore, the gas content) of the galaxies involved. This enhanced star formation then persists through the interaction, even when the other galaxy approaches the apocentre of the interaction. This is seen by the relatively little change in the SFR classifications from stage 2 to 3. While the fraction of MS galaxies does increase slightly, the fraction of starburst galaxies remains essentially constant. There is also no major change in the quiescent fraction either. However, it is interesting to note that the sub-MS fraction does decrease. Thus, the transition to stage 2 to stage 3 of an interaction may bring with it slight enhancement in the SFR, but not on the scale of the other transitions in interaction stage. We then see clear evidence for a second starburst is being driven by the merging of the systems that goes beyond even the enhanced from the initial pericentre passage. 

So, what does the SFR evolution look like from our results? As the interactors approach each other, they are representative of a control sample where there is a population of quiescent, MS and starburst galaxies. However, upon transitioning from stage 1 to 2, the disturbance due to interaction provokes increased fragmentation of a galaxies gas and leads to an increase in star formation. This is true even of the galaxies which have very little gas remaining, as the quiescent fraction reduces significantly in our sample. This enhancement persists from stage 2 to stage 3, where we see no major change in the classification fractions. Then, upon merging and transitioning to stage 4, we find that even the most gas-empty galaxies begin forming stars quickly. 

Now, it is important to note the parameter space that we are searching in these examples. We are probing major interactions between galaxies, where the resultant tidal features that form would be classifiable in an image. It is well know that the mass ratio of an interaction takes an important role in the enhancement of star formation \citep{papers about major interaction}, and we are probing the most extreme examples. Hence, even those mergers which are considered dry - with very little gas - would be capable of driving major rejuvenation in the star formation of these systems. To conclude, \textbf{we find clear, statistically robust evolution of a galaxys' SFR with interaction stage. The SFR changes dramatically from stage 1 to 2 and stage 3 to 4 - at the pericentre and coalescence of the interaction. Existing simulations of galaxy interactions support this interpretation of when star formation changes.}

\subsection{Controlling for Galactic Environment} \label{sec:env-cont}
\noindent While we have found clear evidence of evolution in the SFR of interacting galaxies with stage, it is important we remove a potential source of contamination. It is well known that the environment has a direct impact on the observed SFR of galaxies. A galaxy in a cluster environment has, on average, a higher SFR than those in the field \citep{Papers on environment with SFR}. Thus, if any of our stage classifications are biased towards one environment or another, it could have severe consequences for our results. Therefore, we utilise the existing COSMOS envrionment catalogue described in \citet{2017ApJ...837...16D} to ensure there are no environmental biases in our staging. 

Upon applying the matching described in Section \ref{data:environ}, we break our samples into different environment measures with stage. Figure \ref{fig:dens-stage} shows the distribution of our matched samples with their density values. It is important to note that \citet{2017ApJ...837...16D} has a higher mass cutoff than we have implemented in our underlying sample. Therefore, this is showing the density of all systems $\log$M$_*$/ M$_\odot \geq$ 9.6. 

\begin{figure}
    \centering
    \includegraphics[width=0.45\textwidth]{figures/density-stage.pdf}
    \caption{The density about each of our sources matched with the \citet{2017ApJ...837...16D} catalogue. As shown, there is no existing bias in the distribution of galactic environments throughout our stages. This is confirmed by performing KS- and AD-tests. Therefore, we can conclude that the shown evolution in SFR with interaction stage is not due to environmental effects.}
    \label{fig:dens-stage}
\end{figure}

\section{Nuclear Activity with Interaction Stage}
\subsection{Controlling for Stage}

\section{Characteristics of Matched Pairs}
\subsection{Star Formation and Mass Ratio}

\subsection{Star Formation and Projected Separation}

\section{Limitations of Approach}\label{sec:limitations}
\subsection{Redshift Effects on Visual Classification}
\noindent An important limitation of our visual approach is related to the effect of tidal features out to higher redshifts. As the redshift increases, surface brightness dimming of the systems may lead to a lack of visual confirmation in the tidal features of different objects. Thus, if we were heavily affected by this in our redshift range ($0.0 \leq z \leq 1.2$), we would see distinct signs of it in our selection function of our mergers. For instance, if all of our stage 4 classifications were actually high redshift stage 3 galaxies, we would observe this in our selection function. Therefore, Figure \ref{fig:redshift_selection} shows our selection functions across our redshift range for each defined stage in our interaction.

\begin{figure}
    \centering
    \includegraphics[width=0.45\textwidth]{figures/stage-selection.pdf}
    \caption{Redshift vs Mass distribution for each stage of interaction we have defined. This is important as we use tidal distortion and the existence as tidal features as a fundamental for our classification methodology. As shown here, the distribution of systems across redshift and mass is consistent for all stages in our sample. Therefore, we are likely not affected by this in our analysis.}
    \label{fig:redshift_selection}
\end{figure}

As shown, the distribution of masses across redshift that we find in our sample is reasonably consistent with each other. This means we are likely not affected by surface brightness dimming across our sample and our redshift range.

\subsection{Selection Effects in AGN Sample}

\subsection{Limitations of Visual Classification}

\subsection{Selection Effects of SFR Sample}
 
\section{Conclusions}\label{conclusion}
 - Restate the Results \\
 - What's the next steps of this project?
    - Hint at what I would do in a Fellowship.
 
\section*{Acknowledgements}
The Acknowledgements section is not numbered. Here you can thank helpful
colleagues, acknowledge funding agencies, telescopes and facilities used etc.
Try to keep it short.

This work utilised data from the Cosmic Evolution Survey (COSMOS) which has a digital object identifier of 10.26131/IRSA178.

%%%%%%%%%%%%%%%%%%%%%%%%%%%%%%%%%%%%%%%%%%%%%%%%%%

%%%%%%%%%%%%%%%%%%%% REFERENCES %%%%%%%%%%%%%%%%%%

% The best way to enter references is to use BibTeX:
\bibliographystyle{mnras}
\bibliography{references} % if your bibtex file is called example.bib

%%%%%%%%%%%%%%%%%%%%%%%%%%%%%%%%%%%%%%%%%%%%%%%%%%

%%%%%%%%%%%%%%%%% APPENDICES %%%%%%%%%%%%%%%%%%%%%

\appendix

\section{Needed?}


%%%%%%%%%%%%%%%%%%%%%%%%%%%%%%%%%%%%%%%%%%%%%%%%%%


% Don't change these lines
\bsp	% typesetting comment
\label{lastpage}
\end{document}

% End of mnras_template.tex